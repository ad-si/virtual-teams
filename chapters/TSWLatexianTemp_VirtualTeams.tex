\begin{multicols}{2}
[\section{Virtual Teams}]
In today's age of flourishing globalization, teams are faced with new challenges, one especially important being geographical dispersion of its members. These teams - working across temporal, spatial and maybe also across organizational boundaries - are called virtual teams. Typically they are composed of highly cross-functional members from various ethnical and cultural backgrounds working on highly interdependent tasks. The existence of these teams is enabled solely by the use of information technology, especially communication technology. 

The challenges these teams face are not only those imposed by their goals, but also the management of the virtual interaction. On top of the task-related processes - such as communication and coordination - also socio-emotional processes - such as relationship building, team cohesion and establishment of mutual trust - need to be taken into consideration.\cite{Powell_Piccoli_Ives_2004}\\ Additional strain on the interaction can arise from cultural differences and language barriers. We will discuss the different aspects of the interaction in a virtual team and how it differs from the interaction in a co-located team in the following.
\end{multicols}

\begin{multicols}{2}
[\subsection{Software Development Teams}]
Another type of team to be mentioned is software development teams. These relate to a virtual team insofar, that their day-to-day work includes a lot of the technological tools virtual teams use - in fact they are mostly the teams that create most of the technologies for their own use that other teams are adopting nowadays. As in virtual teams, a lot of collaboration in software development happens asynchronously using these tools. Other than that, software teams are highly specialized teams. Required roles like Architect, Database Administrator, Designer, Programmer, Requirements Analyst, etc. \cite{IBM} are usually assigned to individual team members. 
\end{multicols}