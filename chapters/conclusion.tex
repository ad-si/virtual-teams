\begin{multicols}{2}
[\section{Conclusion}]
In conclusion, we find that working in teams can be of substantial benefit. If in team - be it non-virtual or virtual - members coordinate individual skills on their interdependent task well, each can maximize strengths and minimize weaknesses. Especially diversity, professional or cultural, leads to a multiplication of ideas and opportunities for the team.
Virtual Teams, although profit from the typically very large diversity of their members, face special conditions and challenges. These arise from the novel kind of interaction by means of communication technology as well as the cultural and ethnic differences of its members. However, these challenges can be overcome if team members approach their task with the right attitude and with the help of the technological tools that exists today.
Furthermore we find that the feats required of team members - such as honesty, mutual trust, commitment, reliability, willingness to compromise - are required of virtual and non-virtual teams all the same, the difference solely being the way these are implemented or achieved. Combining these findings with our own experiences, we are convinced that over time working in virtual teams will become just as much part of our work environment as the teamwork perceived as "normal" today.
\end{multicols}