\begin{multicols}{2}
[\section{Definition of a Team}]
Both through the course of our studies and in this lecture we have heard of various attempts at the definition of a team[1,2,3] \cite{katzenbach2003the,BuildingBlocks}. 
For ourselves, we focus the definition on the following three aspects of a team:

  \begin{enumerate}[1.]
	\item A team is composed of at least 3 individuals. 2 people do not form a team, since for a pair of individuals it is not possible to partake in the same rich interactions that are possible in a larger group of people.
	\item A team works towards some common purpose or performance goal. Ideally, the team members commit to a shared vision of what they want to achieve and hold themselves mutually accountable for the work towards this goal.
	\item Lastly, teams can be differentiated from "simple" groups of people through the interdependent nature of their work. In teams, responsibility is shared between members on tasks that require collaboration. Ideally, individual skills can be coordinated so that members are able to maximize their strengths and minimize their weaknesses.
  \end{enumerate}
\end{multicols}