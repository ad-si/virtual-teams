\section{Technology}
\begin{multicols}{2}
One of the key aspects of collaboration is technology. A team is only able to work as good as its software enables it to. This is especially important for virtual teams as software shapes their very core.
There are three key requirements every software has to fulfill \cite{HBR}:

\begin{enumerate}[1.]

	\item Simplicity\\
The more complex a tool is, the more time your team members will need to set it up, learn how to use it and to get started. Time they need to get their job done instead of wasting it  for configuration, integration and incorporation. Besides, more functionality does not necessarily mean better interactions. Your tool should get out of your way and let you focus on what’s really important: the message. One might argue that this wasted time can be saved later, due to a more powerful tool and therefore a more effective work process. However, its likely that you’ll see yourself spending more time on the correct usage of the software than the actual content. So make sure your software does exactly what its supposed to do and does it as precisely and simple as possible.

	\item Reliability\\
One main reason for the ongoing popularity of phone calls in times of advanced software tools and apps like Skype and FaceTime is its reliability. Though it can not compete in terms of functionality, it is without doubt the most unfailing communication tool. The latest video conferencing, instant messaging, or screen sharing feature won’t help you if you’re internet connection can (again) not be established.
Your phone, however, is a reliable and proven communication tool that will not let you down easily.

	\item Accessibility\\
No matter if you’re in Africa enhancing your international reach, in Antarctica figuring out if this is the place for your next server farm or just about anywhere else, you might want to be able to contact your team members at any given time.  Here again your cell phone will serve you well.
\end{enumerate}
\end{multicols}