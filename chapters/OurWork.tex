\begin{multicols}{2}
[\section{Our work as a Virtual Team}]
As mentioned during our presentation, we simulated a virtual team during the creation of this paper and its preceding in-class presentation. In an initial face-to-face kick-off meeting (as described above) we defined our toolset: For asynchronous communication we used Facebook Messenger. We synchronised via Skype as conferencing and screen sharing tool. Google Documents/Drive was our choice for both asynchronous and synchronous collaboration during the planning phase for the presentation and especially in the process of writing this paper. It also includes version control. Finally, we decided on prezi as a presentation software, partly for stylistic reasons, but also because of its real time and collaboration capabilities. We agreed on the English language as language of communication to emulate an environment of team members that have to adapt to a non-native language.
Our experiences during the work on this project were very positive in almost all aspects. Except for the temporary unavailability of the internet connection for one of us our chosen tools proved reliable and simple to use. We also felt entirely comfortable about the collaboration via conferencing and screen sharing. Neither did the result suffer from the virtual work - we did receive a 1.3 for our presentaion. Nor did we feel we were missing out the social aspects, since online messaging and internet calls are as much part of our social life as it was now for the project related work.
This simulation, of course, can only partially simulate a real virtual team environment - we could not test for cultural differences or different time zone activity. However, we believe that virtual interaction - extraordinary and extraordinarily challenging as it might be perceived today - will eventually become just as much part of work as "normal" face-to-face interaction.
\end{multicols}