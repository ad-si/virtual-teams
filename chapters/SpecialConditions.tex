\begin{multicols}{2}
[\section{What special Conditions need to be considered with respect to Virtual Teams?}]
The dissimilarities of the team members – be it by location, by culture etc. – certainly impose the need for special support on the team leader.
We will now explain some of the problems and possible solutions:
\begin{enumerate}[1.]
\item First of all the team members need to be aware that they are working in a virtual team. If they do not keep this premise in mind at all times the team will perform worse than both a collocated team and the average virtual team \cite{outperform}.
\item Geographical distribution
		\begin{enumerate}[a)]
		  \item The fact that team members might be located in different time zones makes planning of virtual meetings or conferences at convenient times difficult. One or more team members might have to work at an inconvenient time and not be capable of perform to their full extent. This problem can either be worked around or even used to the team’s benefit.
			\begin{enumerate}[i.]
			  	\item Plan the meetings at different times so that each team member at least once has the convenience to work normally and the inconvenience to work long hours.
				\item Members who are scattered all over the world can work on a “follow-the-sun” schedule. Whenever a member goes to bed he leaves a to-do list for the next who gets up and can then start working on these tasks. Thus, the work on the project is never halted.
			\end{enumerate}
		  \item Due to the possibly scattered locations meeting the team members in person can be difficult for reasons of price and time-effectiveness. If the duration of the project is long enough the effort will most certainly have positive effects and make up for the initial investment by boosting the team’s performance and spirit. See ‘The kick-off meeting’.
		\end{enumerate}
\item Cultural differences
		\begin{enumerate}[a)]
		  \item Social habits \\ In different cultures there are different ways of beginning and leading a conversation. Americans often use the set phrase “How are you?” upon which Germans would like to answer with their actual situation. The other way around the American might be offended by the German not asking for his well-being. Although these are only very small issues they can - in sum - lead to tension among the team members and need to be taken care of beforehand.
		  	\item Way of discussing \\ Especially in the Arabic world it is common to discuss very hotly. However other cultures are usually not used to this behaviour and might easily misinterpret the way of discussion for aggression and therefore get angry themselves. The team  manager must assure that either the temper of the one is restrained or the others are made aware to the fact that their counterpart is just discussing normally.
		\end{enumerate}
\item Language
		\begin{enumerate}[a)]
		  \item Knowledge of English language prerequisite \\
As it is the ‘business language’ being business fluent in English is a requirement for working in an international team. Unfortunately machine translation is not yet advanced far enough to make the actual knowledge of a common language obsolete.
		  \item Translations/idioms/’false friends’\\
If not all members can resort to a common language the lack of mutual understanding can lead to conflicts within the team. Another problem is idiomatic speech that is not known to everybody, especially if sayings are translated directly into English e.g. “Vom Regen in die Traufe” directly translates to “From rain to eaves” but rather corresponds to “from the frying pan into the fire”.
		  \item Technical terms in different language. Only know the terms referring to your field of expertise. \\
It is a common thing to know the terms of one’s field of expertise in English but less so with the vocabulary of other fields. Explaining the terms of software engineering to a designer can cost precious time. In such cases the team leader needs to assess the necessity of such explanations.
		\end{enumerate}
\end{enumerate}
\end{multicols}