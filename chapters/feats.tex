\begin{multicols}{2}
[\section{Feats of your team members}]
The team members can be diverse in many different ways. Apart from being in different locations or speaking different languages they have different skills and professional backgrounds. In order to profit from this diversity all team members need to embrace the idea of being different and dispel biases against other fields of expertise. As Kathleen Eisenhardt points out two engineers make the worst team. Through diverse views on things more options are available from which to choose the best suited one.

Communication is another key aspect of members working efficiently in a team. In general teams of introverts who tend to communicate less  produce worse results than teams of extroverts. Through extensive communication among the team members they will create more cohesion, work more coordinately and perform better. Through communication about more personal topics help build relationships between the members of a team. If two persons dislike each other they can omit the personal talk and focus on the topic-related issues instead.
Good communication on the task leads to more efficiency by having more iterations of question-answer-cycles and helps to avoid a doubling of work.

Being honest is a prerequisite to being communicative about all aspects of the work progress including advances but also delays, problems and failure. The members need to be informed about such issues to help solve them faster. It may seem easier to deceive others in a virtual environment but it turns out that in fact e-mail is a more ‘honest’ media than most other including telephone and speech \cite{TED}.

Communicating openly will eventually lead to trust among the team. Trusting the other team members and the quality of their contributions is a basis for delegating tasks. If they don’t trust one another they might settle for doing the tasks themselves creating doubling and wasted efforts.

Every member needs to contribute at least a basic set of skills to the team. They must quite naturally be in the specific field of expertise. Introducing measures for estimating the person’s performance can be a good mean but needs to be applied carefully. Additionally, social skills matter just as much as the technical ones and are the key to  successful teamwork. [Guy Kawasaki].
Particularly over long projects the team manager must seek to maintain the motivation of all team members. In order to do this success and accomplishments of individual team members must be praised. A pat on the shoulder still remains a great motivator. Having in mind that the contribution is appreciated can also lead to additional individual commitment  for the project.

Insisting on your own opinion will not contribute to the team’s success. In particular if there are many different fields of expertise present this will lead to a wide variety of opinions. Before you get stuck in endless debates you must make a compromise. Find a solution that everybody can be content with.
\end{multicols}