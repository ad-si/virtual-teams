\begin{multicols}{2}
[\section{10 “Must Have” Software Tools}]
The following list will enumerate the 10 “must have” tools every team should be using. It covers the fundamental collaboration techniques and helps your team to gain efficiency. Furthermore, it will give you an extensive insight on what modern technology is capable of and provides a good reference point for your custom software solution.
\begin{enumerate}[1.]
  \item Conferencing Software\\
The most basic and probably most important type of communication is talking to your team members, via phone, the internet or in person. To enhance the limited interaction of a phone call one can use conferencing software. This is a tool that enables you to talk to several people simultaneously. Video conferencing software even provides the possibility to see each other face to face and thereby provides a more personal setting. Due to further enhancements like instant messaging and data transmission this software can satisfy large parts of your requirements for collaboration.
  \item Screen Sharing\\
Closely related to conferencing is screen sharing software. Additionally to seeing and hearing your partner you might want to be able to see what he’s working on right now. Instead of having to save and share his progress laboriously he can easily share his screen. This enables you to see what he’s doing in real time. That brings us to our next point.
	\item Real time collaboration\\
The two previous tools already were examples of real time tools. Real time means that you can collaborate simultaneously and synchronously on the same task. I.e. it is very useful to work on a document simultaneously as you can see what the others are writing and you don’t have to merge it intricately afterwards. Another example is pair programming, which is considered to be extremely beneficial.# Writing code simultaneously together means that it will be high-quality, include knowledge from different domains and that the general work atmosphere is much more intense.
	\item Version Control\\
Version control describes the method of saving your works progress at certain points in time. This can be archived by simply using the “save as” functionality of your program to capture improvements in the documents with snapshots or by using a more sophisticated version control system like git.
This will even allow you to revert documents to an older state, compare different versions, merge them or to get detailed insights in the development process. Furthermore it servers as your backup utility as you can easily push your changes on different servers or to team members to let them inspect your work and use your documents and your data.
	\item Issue tracking\\
To keep track of all the things your team has to do, make sure to set up a issue tracking program. It provides a fast and easy interface to manage all your tasks. There are several aspects that will improve your overview and productivity dramatically.
		\begin{enumerate}[a)]
		  \item Assignment\\
Assign tasks to certain team members or sub teams to clearly divide and define the teams responsibilities and accountabilities. 
		  \item Time Tracking\\
Define the deadlines and group the issues to milestones to clearly outline your further steps. In addition monitor the implementation time of features and by that the efficiency and productivity of your team members.
		  \item Type Definition\\
By specifying the type of your issue, i.e. bug, feature, improvement, support, etc. you have a very useful measure of your work quality. The number of bugs for example gives insights on how accurate the work is done and the number of features is an indicator of the innovative capacity of your workforce.
		\end{enumerate}
	\item Calendar\\
As already mentioned it is important to capture your goals and tasks for the future. Very handy for this purpose is a globally shared calendar. It’s the place where all you team mates can store and look up what’s going on and what will be important in the near future. Ideally all your other software tools also have access to this calendar so that everything is neatly kept at one place.
	\item Human Resource Management\\
Another very important aspect of collaboration is that you should know exactly who you are working with to be able to choose the best for every task. That’s why you want to have software to manage all your team members. A short profile page will tell what a person has been doing before, what expertise he is contributing to the team and what his plans and dreams for the future look like.
It can even give you a quick overview on what that person has already achieved in the team and how much time he or she has spent on work. Thereby you can more easily delimit performers from non performers and build your perfect team with just the right people.
	\item Forums\\
Conferencing software is great for synchronous communication. Sometimes, however, you might not be able to communicate synchronously due to time shifts or a general shift in work time. Occasionally you will also want to make sure that everything you are talking about is written down so that nobody can claim to have said something else than he actually did. That’s when a forum comes in handy.
It constitutes a simple but yet effective means to discuss different ideas and topics.
	\item Wiki\\
Wikis are the tool of choice when it comes to any kind of documentation. It’s the place where you should write down your communication guidelines, your style guidelines, software documentation or just about anything else that needs to be remembered and be accessible for everyone.
	\item Monitoring\\
In order to keep your team up to date it's very useful to set up a central monitoring website. There you can display graphs about market share, market penetration, customer satisfaction, visitor numbers etc. It can also contain news feeds and widgets to your other software tools. This site is very important as it normally is your central drop-in center and start screen for your following work. Also consider to put up a screen at a central place in your office so that everybody can look up the latest numbers with a glimpse.
\end{enumerate}

While using the tools it is essential to review your productivity on a regular basis and if applicable rethink your software setup. Try to find your favorite toolset!
\end{multicols}